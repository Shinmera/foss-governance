\section{Introduction}
Most people have, somewhere in their life, encountered free and open source software (FOSS). But how is it possible that such projects can persist, even though most of the time they aren't controlled by huge companies planning their development? How is it possible for a community to come together and create something new? Who is interested in the development of free software? How is the quality of the final product assured? We explore these questions by investigating a couple of projects, both small and large, with more and less corporate involvement. \\

We have selected the projects based on criteria of relevance, relative size, commercial or corporate involvement, activity, and our own personal attachment. Linux and Clang were chosen due to their large size and activity, and the relatively strong commercial involvement. The difference between the two stems mainly from their age and origin. Blender, while smaller than the other two, is a still relatively large project, but is aimed at a very different crowd. PBRT, OpenVDB, and Clasp were chosen due to our respective personal interest and involvement in the projects. All of the projects each make use of a different license, form of community management, and development strategy, which should lead to a good, general overview of development in FOSS communities. \\

We follow with a close analysis of each project, then draw parallels and common aspects between them, finally leading to a conclusion that explores the general tendencies, benefits, and drawbacks of the different models and approaches that we encountered.

%%% Local Variables:
%%% mode: luatex
%%% TeX-master: "foss-governance"
%%% End:
