\subsection{Clasp}{Nicolas Hafner}
Clasp\cite{clasp-github} is an implementation of the Common Lisp language standard (ANSI INCITS 226-1994). It is a rather small project in terms of contributors, with only a few select developers actively working on it at the time of writing. \\

Clasp started in Summer 2012\cite{clasp-timeline}, in an effort to create a language implementation that would be suitable for high-level molecular design, while at the same time being able to retain high performance, and access to a large ecosystem of C++ chemistry libraries\cite{clasp-cando}. Common Lisp was chosen due to its high amount of dynamic and flexible language features, as well as its ability to give access to detailed optimisation strategies when necessary\cite{clasp-talk}. \\

The project remained a private endeavour for a long time, until it was published on GitHub in May 2014\cite{clasp-github}. The prospect of a Lisp implementation that could promise seamless interoperation with C++ caused quite a bit of attention upon its release. However, due to frequent issues building the project, lack of documentation, and a long line of bugs, contribution from people outside of Christian Schafmeister himself did not really happen for a good while. \\

Currently the project is still most actively developed by Schafmeister himself, although a few more people have started to contribute as well. Clasp now uses the Cleavir compiler by Robert Strandh\cite{clasp-cleavir} as its final, end-user compiler. Cleavir itself is still under very active development. Strandh and Schafmeister often discuss the future of the two projects together, leading to a close collaboration between the two. While Cleavir in itself, too, is mostly a single-person effort, it recently has also received work from other people. Clasp also makes use of the Ravenbrook Memory Pool System (MPS)\cite{clasp-mps} garbage collector. This, too, has prompted collaboration between Schafmeister and Ravenbrook, in order to improve the MPS for more dynamic and special environments. \\

The primary source of funding for the project has come from Schafmeister's occupation as a professor at Temple University. It has also been in part funded by various grants supplied by the US government\cite{clasp-grants}. Due to the esoteric and ambitious nature of Clasp, getting access to enough funding has been a continuous problem however. A few companies have expressed interest in Clasp, though only one of them (Configura Inc.) has contributed financially so far.

%%% Local Variables:
%%% mode: latex
%%% TeX-master: "foss-governance"
%%% End:
