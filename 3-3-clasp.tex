\subsection{Clasp}{Nicolas Hafner}
Clasp\cite{clasp-github} is an implementation of the Common Lisp language standard (ANSI INCITS 226-1994). It is a rather small project in terms of contributors, with only a few select developers actively working on it at the time of writing. \\

Clasp started in Summer 2012\cite{clasp-timeline}, in an effort to create a language implementation that would be suitable for high-level molecular design, while at the same time being able to retain high performance and access to a large ecosystem of C++ chemistry libraries\cite{clasp-cando}. Common Lisp was chosen due to its high amount of dynamic and flexible language features, as well as its ability to give access to detailed optimisation strategies when necessary\cite{clasp-talk}. \\

The project remained a private endeavour for a long time, until it was published on GitHub in May 2014\cite{clasp-github}. The prospect of a Lisp implementation that could promise seamless interoperation with C++ caused quite a bit of attention, upon its release. However, due to frequent issues building the project, lack of documentation, and a long line of bugs, contribution from people aside from Christian Schafmeister himself did not really happen for a good while. \\

Currently the project is still most actively developed by Schafmeister himself, although a few more people have started to contribute directly as well. Some have started writing code in order to provide more independent systems, or to clean up the code base. Others are helping the project by testing it, reporting bugs, and managing the issues. \\

When looking at projects Clasp makes use of, things are a bit more intriguing. Clasp now uses the Cleavir compiler by Robert Strandh\cite{clasp-cleavir} as its final, end-user compiler. Cleavir itself is still under very active development. Strandh and Schafmeister often discuss the future and current issues of the two projects together, leading to a close collaboration between the two. While Cleavir in itself, too, is mostly a single-person effort, it recently has also received work from other people. A large part of the Clasp source tree is also taken from the Embeddable Common Lisp (ECL)\cite{clasp-ecl} project and contributions to that code-base are ported over to Clasp as well. Clasp also makes use of the Ravenbrook Memory Pool System (MPS)\cite{clasp-mps} garbage collector. This has prompted collaboration between Schafmeister and Ravenbrook, in order to improve the MPS for more dynamic and special environments. Clasp also makes use of very new and untested features of the LLVM and Clang systems in order to improve the speed and interoperability of Clasp. This has resulted in multiple direct exchanges with engineers from Apple and LLVM in order to fix issues in LLVM that affected Clasp. \\

Thus the contribution situation for Clasp is of an interesting nature. It makes heavy, direct use of several projects like LLVM, Cleavir, ECL, and MPS, which prompts for collaboration between the two, leading to improvements and changes on both sides. This is unlike most typical software projects, where libraries and frameworks are simply made use of, without any communication between the project and the library maintainers. Clasp's situation is mostly due to the immaturity of the libraries it makes use of, as they are still lacking in features or expose bugs that Clasp encounters due to its high demands. \\

The primary source of funding for the project has come from Schafmeister's occupation as a professor at Temple University. It has also been in part funded by various grants supplied by the US government\cite{clasp-grants}. Due to the esoteric and ambitious nature of Clasp, getting access to enough funding has been a continuous problem however. A few companies have expressed interest in Clasp, though only one of them (Configura Inc.) has contributed financially so far. \\

A large part of the collaborative activity of Clasp happens on the Freenode IRC channel \#clasp. Developments and problems are discussed between various people there, even those that do not contribute to Clasp in code. Frequently issues are addressed there directly, though depending on the severity of it they may get elevated and registered as a GitHub issue ticket. Most of the time final decisions are still felled by Schafmeister himself, though he is also frequently driven in new directions by people that display a large amount of knowledge and insight into compiler technology and Common Lisp development. \\

In conclusion, something something.

%%% Local Variables:
%%% mode: latex
%%% TeX-master: "foss-governance"
%%% End:
