\subsection{Blender}{Felice Serena}

% intro
The goal of Blender's community is to create, maintain and use a ``free and open source 3D creation suite'' that ``supports the entirety of the 3D pipeline — modeling, rigging, animation, simulation, rendering, compositing and motion tracking, even video editing and game creation''. 
The largest forum\cite{blender-forum} has over 200'000 registered members, and more than a hundred different developers have committed code to the project. \\

% community structure
The community is led by Ton Roosendaal who is chairman of the Blender Foundation. 
He founded the Blender Institute in 2007. 
It's a permanent office and studio to improve Blender's development.\cite{blender-foundation-history}
The Blender Foundation pays a small team of full-time core developers to maintain Blender.\cite{blender-development-support}\footnote{
One might be confused about the difference: The Blender Foundation is the organisation that finances all Blender related costs. 
The Blender Institute is a permanent location to give office space for large Blender related projects.
}
According to Blender's Homepage\cite{blender-homepage}, the community consists of ``studios and individual artists, professionals and hobbyists, scientists, students, VFX experts, animators, game artists, modders, [and more]''.
Artists and studios most probably use Blender for creating art, examples are still lifes, movies, VFX, games, and 3D printing.
To scientists, Blender allows manipulating their 3D scans. 
A group of archaeologists used Blender to perform a forensic facial reconstruction.\cite{blender-facial-reconstruction} \\

% history
Ton Roosendaal is since 1995 the driving force behind the vision of a free software solution for the 3D pipeline. 
After having founded the animation studio NeoGeo in 1988\cite{blender-manual-history}, he and his team intended to rewrite their in-house software. 
He founded Not a Number (NaN) to develop Blender. 
NaN's business model was to provide a free 3D suite, around which they would provide services and commercial products. 
After being a success on SIGGRAPH 2000, over €4.5M have been invested by venture capitalists. 
Though having a user community over 250'000, NaN's investors decided to shut down the project due to disappointing sales and a changed economic climate. 
2002 the Blender Foundation emerged driven by Ton Roosendaal and the community, they bought the rights on Blender from the investors for €100'000 and released it under GNU GPL\cite{blender-license} in October 2002. 
The goal of the Blender Foundation is to ``find a way to continue developing and promoting Blender as a community-based open source project.'' \cite{blender-foundation-history} \\

Since 2002 more than 3'000'000 lines of code have been changed in over 66'000 commits (24.11.2016) by 149 different contributors.\cite{blender-repository} 
During that time Blender has been extended in many fields (simulation, motion tracking, ray tracing, 3D printing, major UI redesign, …). \\

% money
Blender is mainly financed by the Blender Foundation who's goal it is to find ways to finance Blender.
The Blender Foundation achieves this goal with a online store\cite{blender-estore}, the organisation of the yearly Blender Conference\cite{blender-conference}, the Blender Network\cite{blender-network}, the Blender Cloud\cite{blender-cloud}, crowd foundings, donations\cite{blender-donate}. \\

% conflicts
Blender is like a huge tool box. 
Most people just use what the need and can easily ignore parts of the software they don't need since the GUI is very customizable and flexible. 
Though, there's one issue that arises on a regular basis. 
Blender hasn't the easiest interface. 
Its developers always focused on flexibility, not simplicity.
Novice users often experience quite a challenge in learning Blender's user interface.
Professional users appreciate the GUI's complexity since they say it helps them to be more productive.
There have been improvements in this area (as with the release of Blender 2.5), but in the year 2013, the situation escalated when Andrew Price led a movement to improve Blender's user interface and make it easier to understand.\cite{blender-guru-ui} 
Ton Roosendaal wrote as answer a blog-post\cite{blender-ton-ui} where he explained in great detail that Blender is intended for professional users and one shouldn't confuse this with the fact that it's accessible for everyone.
His clear statement stopped any further discussion. \\

% code maintenance
Blender's source code is stored in the form of a git repository on blender.org\cite{blender-repository}.
Blender is subdivided into multiple modules; each has a developer assigned, a so-called module owner.\cite{blender-module-owners} A module owner decides on design and implementation issues, and approves or rejects patches and feature requests.\cite{blender-module-owner} 
An owner has to coordinate work with other teams.
Most parts of the Blender source code are written in C (56\%), C++ (27\%), and Python (14\%).\cite{blender-code-stat} \\

% new features
New ideas and feature requests are normally first discussed and filtered by the community in forums or other channels like blogs, podcasts or brain-storming websites.\cite{blender-community} 
If something sounds like a good new idea and a developer can be found, the owner of the corresponding module of Blender is then contacted to check if the idea conforms to Blender's goals. 
On a regular basis, Blender participates in Google Summer of Code where students are encouraged to work on Open Source software development.\cite{blender-gsoc-2016}
After the requirements are set, the developer posts a design document on the wiki, describing what exactly he/she intends to implement.\cite{blender-new-devs}
Things like coding style\cite{blender-style-rules} and commit rules\cite{blender-new-devs} are well documented in Blender's wiki. 
Conforming to all guidelines, the new feature is then implemented and added to the main trunk in the form of patches. 
The new feature then develops regularly with Blender's release cycle.

%%% Local Variables:
%%% mode: latex
%%% TeX-master: "foss-governance"
%%% End:
