\subsection{Blender}{Felice Serena}

% history
The goal of Blender's community is to create, maintain and use a ``free and open source 3D creation suite'' that ``supports the entirety of the 3D pipeline — modeling, rigging, animation, simulation, rendering, compositing and motion tracking, even video editing and game creation''. The largest forum\cite{blender-forum} has over 200'000 registered members, and more than a hundred different developers have committed code to the project. Ton Roosendaal has been the driving force behind the vision of a free software solution for the 3D pipeline since 1995. After having founded the animation studio NeoGeo in 1988\cite{blender-manual-history}, Ton Roosendaal and his team intended to rewrite their in-house software. To do so, he founded NaN (Not a Number) to develop Blender. \\

NaN's business model was to provide a free 3D suite around which they would provide services and commercial products. Thanks to a successful presentation at SIGGRAPH 2000, over €4.5M have been invested by venture capitalists. Despite a user community of over 250'000 people, NaN's investors later decided to shut down the project due to disappointing sales and a changed economic climate. 2002 the Blender Foundation emerged driven by Ton Roosendaal and the community, they bought the rights for Blender from the investors for €100'000 and released it under GNU GPL\cite{blender-license} in October 2002. The goal of the Blender Foundation was then set to ``find a way to continue developing and promoting Blender as a community-based open source project.'' \cite{blender-official-history} \\

Since 2002 more than 3'000'000 lines of code have been changed in over 66'000 commits (24.11.2016) by 149 different contributors. In that time, Blender has been extended to many fields beyond 3D modelling such as simulation, motion tracking, ray tracing, 3D printing, etc. \\

% community structure
The community is led by Ton Roosendaal, who is chairman of the Blender Foundation. He founded the Blender Insitute in 2007. It is a permanent office and studio, with the aim to improve Blender's development.\cite{blender-foundation-history} The Blender Foundation pays a small team of full-time core developers to maintain Blender.\cite{blender-development-support} According to Blender's Homepage\cite{blender-homepage}, the community consists of ``studios and individual artists, professionals and hobbyists, scientists, students, VFX experts, animators, game artists, modders, [and more]''. Artists and studios most likely use Blender for creating 3D models, movies, visual effects, games, and 3D printed models. To scientists, Blender allows manipulating their 3D scans. On the more extravagant side, a group of archaeologists used Blender to perform a forensic facial reconstruction.\cite{blender-facial-reconstruction} Blender regularly participates in Google Summer of Code events, where students are encouraged to work on Open Source software development.\cite{blender-gsoc-2016} \\

% money
Blender is mainly financed by the Blender Foundation. The Blender Foundation achieves this goal with an online store\cite{blender-estore} of merchandise, the organisation of the yearly Blender Conference\cite{blender-conference}, the Blender Network\cite{blender-network}, the Blender Cloud\cite{blender-cloud}, crowd foundings, and donations\cite{blender-donate}. \\

% conflicts
Blender is like a huge tool box. Most people just use what they need and can easily ignore the rest, as the user interface is very customizable and flexible. However, one issue arises on a regular basis. Blender's interface has always been very pragmatic, addressing professional artists. With version 2.5, where the entire interface was redesigned, this problem was fixed for the most part, but there is still a steep training curve for novices. This conflict escalated in 2013, when Andrew Price led a movement to improve Blender's user interface and make it easier to understand.\cite{blender-guru-ui} Ton Roosendaal ended the discussion by making it clear what the intended goals and priorities should be.\cite{blender-ton-ui} This shows that Ton Roosendaal has the last say on Blender's development. \\

% code maintenance
Blender's source code is stored in a git repository on blender.org\cite{blender-repository}. Blender is divided into multiple modules, each of which have a core developer assigned.\cite{blender-module-owners} Most parts of the Blender source code are written in C (56\%), C++ (27\%), and Python (14\%).\cite{blender-code-stat} \\

% new features
New ideas and feature requests are usually first discussed and filtered by the community in forums, blogs, podcasts, or brain-storming websites.\cite{blender-community} If something sounds like a good idea, and a developer can be found, the owner of the corresponding module of Blender is contacted to check if the idea conforms with Blender's goals. After the requirements are set, the developer posts a design document on the wiki describing what exactly he/she intends to implement.\cite{blender-new-devs} Things like coding style\cite{blender-style-rules} and commit rules\cite{blender-new-devs} are well documented on Blender's wiki. Conforming to all guidelines, the new feature is then implemented and added to the main trunk in the form of patches. From there on out, the feature is then developed regularly with Blender's release cycle.

%%% Local Variables:
%%% mode: latex
%%% TeX-master: "foss-governance"
%%% End:
