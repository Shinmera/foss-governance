\subsection{Linux}
With almost 14'000 submitted patches, around 1'600 developers, and about 270 companies contributing to every release\cite{linux-whowrites}, the Linux Kernel project is one of the largest collaborative efforts to date. Its long history, reaching as far back as 1991\cite{linux-announce}, and its massive success with being installed on over 1.3 billion devices\cite{linux-usage} today, makes it a good target to analyse Open Source development and community management on a large-scale basis. \\

Linux started out as a small endeavour by Linus Torvalds to create an operating system that could make full use of his new machine. He modelled it strongly after Minix, but did not use any of its code, in order to avoid inducing potential license fees. Another strong motivation for Torvalds was to do it ``just for fun''\cite{linus-biography,linux-history}. \\

Not much information could be found about how contributions to the Linux project were handled after its initial 0.0.1 release announcement on Usenet. It is fair to assume however, that contributions were mainly handled exchanging patches via e-mail either between interested parties, or Linus directly. This assumption stems mainly from how the Kernel project is still largely being managed today, which I will discuss later. As Torvalds managed the primary FTP server where Linux could be obtained at the time, it is likely that most patches that were of interest reached Torvalds in some manner, who then got to decide whether to incorporate them. If incorporated, they would be present in the next source release, which other people could download again. It seems that Torvalds thus naturally emerged as a form of leader and safe-keeper to the project's development. After some years of independent development, Torvalds was hired by Transmeta Corporation\cite{transmeta} to continue development on Linux. He later switched to the Open Source Development Lab (now The Linux Foundation) and has remained there until today\cite{linus-biography,linus-osdl}. \\

How contributions and discussion is managed is described in detail in \cite{linux-participation}. I will outline the details that are relevant to our paper here. At the time of writing, the discussion and development of Linux primarily happens on the Linux Kernel Mailing List (LKML) and other, closely associated lists. While each developer will keep a local copy to work on Linux independently, actual contributions are only accepted in the form of patches attached to an email to the mailing list(s). As such, for each issue and contribution, a new email thread is started, to which interested parties can respond. There is however no guarantee that anyone will respond, as the process depends entirely on the whims of the people reading the mailing lists. The review of a patch is handled by mail as well, wherein people re-post the patch to the list with an added ``Acked-By'' signature added. Feedback and corrections happen in a similar manner. \\

In order to help reduce and filter the amount of correspondence going on, the Linux Kernel is divided up into several ``subsystems'' that are responsible for smaller or larger components of the Kernel itself. Each subsystem has its own maintainer and its own mailing list. The maintainer is responsible for gathering and reviewing potential contributions, and sending them on to Torvalds. Torvalds then has the final say on whether a patch makes it into the Linux mainline or not. Depending on the size of the subsystem, the primary maintainer may have appointed several sub-maintainers in order to divide up the work necessary to manage all the activity. After some weeks of development, Torvalds then pulls in all the patches from the maintainers of the subsystems and prepares a release candidate version. Following that, only fixes and revisions are accepted, which again go through multiple release candidate iterations, until finally a full release is produced. \\


%%% Local Variables:
%%% mode: latex
%%% TeX-master: "foss-governance"
%%% End:
