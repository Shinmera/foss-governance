\subsection{Linux}{Nicolas Hafner}

With almost 14'000 submitted patches, around 1'600 developers, and about 270 companies contributing to every release\cite{linux-whowrites}, the Linux Kernel project is one of the largest collaborative efforts to date. Its long history, reaching as far back as 1991\cite{linux-announce}, and its massive success with being installed on over 1.3 billion devices\cite{linux-usage} today, makes it a good target to analyse Open Source development and community management on a large-scale basis. \\

Linux started out as a small endeavour by Linus Torvalds to create an operating system that could make full use of his new machine. He modelled it strongly after Minix, but did not use any of its code, in order to avoid inducing potential license fees. Another strong motivation for Torvalds was to do it ``just for fun''.\cite{linus-biography,linux-history} \\

Not much information could be found about how contributions to the Linux project were handled after its initial 0.0.1 release announcement on Usenet. It is fair to assume however, that contributions were mainly handled by exchanging patches via e-mail either between interested parties, or Torvalds directly. This assumption stems mainly from how the Kernel project is still largely being managed today, which we will discuss later. As Torvalds managed the primary FTP server where Linux could be obtained at the time, it is likely that most patches that were of interest reached Torvalds in some manner, who then got to decide whether to incorporate them. If incorporated, they would be present in the next source release, which other people could download again. It seems that Torvalds thus naturally emerged as a form of leader and safe-keeper to the project's development. After some years of independent development, Torvalds was hired by Transmeta Corporation\cite{transmeta} to continue development on Linux. He later switched to the Open Source Development Lab (now The Linux Foundation) and has remained there until today.\cite{linus-biography,linus-osdl} \\
% FIXME: add license info
How contributions and discussion is managed is described in detail in \cite{linux-participation}. We will outline the specifics that are relevant to our paper here. At the time of writing, the discussion and development of Linux primarily happens on the Linux Kernel Mailing List (LKML) and other, closely associated lists. While each developer will keep a local copy to work on Linux independently, actual contributions are only accepted in the form of patches attached to an email to the mailing list(s). As such, for each issue and contribution, a new email thread is started, to which interested parties can respond. There is however no guarantee that anyone will respond, as the process depends entirely on the whims of the people reading the mailing lists. The review of a patch is handled by mail as well, wherein people re-post the patch to the list with an added signature. Feedback and corrections happen in a similar manner. \\

In order to help reduce and filter the amount of correspondence going on, the Linux Kernel is divided up into several ``subsystems'' that are responsible for smaller or larger components of the Kernel itself. Each subsystem has its own maintainer and its own mailing list. The maintainer is responsible for gathering and reviewing potential contributions, and sending them on to Torvalds. Torvalds then has the final say on whether a patch makes it into the Linux mainline or not. Depending on the size of the subsystem, the primary maintainer may have appointed several sub-maintainers in order to divide up the work necessary to manage all the activity. Torvalds then decides on the release schedule and which patches finally make it into the release. \\

Thus the entire system is very heavily reliant on the dynamic decision of individuals and is aside from the very loose and short hierarchy of maintainers mostly self-governing. This begs the question as to how conflicts are actually resolved. According to \citet{linux-participation}, conflict is resolved between individual participants of a discussion, and any attempt at arguing outside of the technical context is frowned upon. Only if a conflict cannot be resolved, or escalates further, other members of the community are actively sought out. Whom to contact for this seems to mostly depend on their standing within the community and how respected their are, where the respect is earned through long standing contributions. Contacting Torvalds is seen as a last-resort tactic. \\

Corporate involvement was present in the Linux kernel project from a rather early age on\cite{linus-biography}. The primary focus at first was to get Linux to support the commercial hardware the vendors sold, as they realised that Linux could pose a good opportunity to drop expensive in-house operating systems. Another involvement later was to hire Torvalds to work on Linux full-time. Since then, company involvement has increased massively as Linux has become the de-facto standard operating system on servers\cite{linux-server} and mobile devices\cite{linux-usage}. Companies primarily get involved by either supporting the Linux Foundation through donations\cite{linux-foundation}, or by hiring developers to actively work on the kernel\cite{linux-whowrites}. Now the features added by corporations not only focus on hardware driver support, but also on speed and infrastructure improvements in order to lower costs in data centres\cite{linux-bbr}. \\

Corporations do not however get any primary say in how Linux evolves. There is no board of directors or a committee that decides how things are run. Everything is still handled by the community, and changes to the kernel often have to come from those who desire them. Despite the aforementioned corporate support, the main contributing factor to Linux' development still comes from independent individuals, who participate in Linux out of their own will. A study performed in 2000 by \citet{linux-motivation} showed that one of the primary motivational factors for people to participate is an increased sense of involvement with the linux community at large, as well as the drive to improve software they use themselves, or to improve their career. Surprisingly enough, perceived trust within a subsystem seemed to only play a minor role, indicating that many may not fear exploitation of their own efforts for the project. This is contrary to frequent debates in OSS, wherein many express fears of their efforts going to waste if they ``work for free''. \\

In conclusion, despite the nowadays heavy corporate investment in the Linux kernel project, it is still primarily governed by a community of individuals and volunteers. Financial support for the project comes mainly out of the pockets of the supporters themselves who contribute out of their own volition, or from companies that support the Linux Foundation monetarily, as they benefit from improvements to the kernel. The project was initially created to provide a free alternative operating system and has since grown into a mature, widely-adopted system with a huge community surrounding it.

%%% Local Variables:
%%% mode: latex
%%% TeX-master: "foss-governance"
%%% End:
