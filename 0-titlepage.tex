\thispagestyle{empty}
\begin{center}
  {\bfseries\Huge\doctitle} \\
  \vspace{4cm}
  \parbox{0.7\linewidth}{\small
    Free/Open Source projects are often self-organised undertakings with an online presence. But how are lots of volunteers, who often don't even know each other, supposed to be able to create functioning software? \\
    
    The central question is how FOSS projects organise themselves. We research three big, well-known projects, and three small, lesser-known projects and elaborate how they work. In doing so, we focus on questions of community building and management, legal status and foundation, code management, and so forth. We also explore how different parties of interest get along and resolve potential conflicts.} \\
  \vspace{7cm}
  Authored by \\
  Nicolas Hafner, D-INFK, hafnern@student.ethz.ch \\
  Felice Serena, D-INFK,  fserena@student.ethz.ch \\
  Delio Vicini, D-INFK, dvicini@student.ethz.ch \\
  \vspace{1cm}
  This report was created as part of the lecture \\
  ``Digitale Nachhaltigkeit in der Wissensgesellschaft'' by Dr. Marcus M. Dapp. \\
  \vspace{1cm}
  The article can be used under the CC-BY license. \\
  \url{https://creativecommons.org/licenses/by/4.0/} \\
  \vspace{2cm}
  Fall semester 2016 \\
  ETH Zürich
\end{center}

\newpage

%%% Local Variables:
%%% mode: latex
%%% TeX-master: "foss-governance"
%%% End:
