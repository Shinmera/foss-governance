\section{Conclusion}
With the information gathered from our case studies, we can now attempt to answer the questions we posed ourselves in the introduction. \\

Open-source software projects generally seem to be started by a group or a single individual that is capable of performing most of the work to get an initial prototype off the ground. From there on out, thanks to the capabilities of the internet to allow cost-effective and rapid sharing, the project can be distributed to other people all over the world. Interested parties can then decide to contribute back. \\

The assurance of quality and the organisation of a project are realised by rather flat hierarchies of involved people, whose standing is rather dynamically decided upon by the community itself. Leaders emerge due to their respective contributions and interactions with the rest of the community. In particular, quality of code is enforced by a required consensus, where patches are only allowed into the mainline distribution of the project should responsible parties agree with it. \\

However, thanks to the permission to modify and redistribute in open source licenses, even in the case of disagreements, development is not at a standstill. Individuals can create forks of a project and drive it ahead in whatever way they see fit. Since the bulk of contributions comes from people who are doing it out of their own volition, it is also practically impossible to enforce your own ideas upon others and the project, lest you do the work yourself. \\

Despite this, the lack of a unified vision for the project can also lead to an irregularity in terms of quality and presentation. As there is no controlling agency that can ensure coherence and decide on a direction, different parts of a project can be in wildly different conditions. \\

As such, open source software vastly benefits from its fluent and free model of community involvement in that it makes it difficult for any individual's effort to go to waste, and in that it can exploit the vast number of interested parties and developers in order to create a potentially much more capable driving force than a corporation could be. However, at the same time this openness can be a detriment, as, despite the loose models of hierarchy and quality-checking, the visions of the individual contributors can differ vastly, leading to a lack of unity in design and presentation overall.

%%% Local Variables:
%%% mode: latex
%%% TeX-master: "foss-governance"
%%% End:
