\subsection{Clang}{Delio Vicini}

Clang is an open-source compiler for C, C++, Objective-C, Objective-C++ and CUDA. The Clang compiler project was started by Apple in 2005 in order to replace the GNU Compiler Collection (GCC). The original motivation for the Clang project was to have a fast, modular compiler which is easier to extend
than GCC and also integrates well with integrated development environments (IDEs).\cite{clang-motivation} A tight integration of the compiler with the IDE allows to for example to support code autocomplete directly using the data structures used by the compiler. Technically, the Clang compiler builds on the LLVM\footnote{LLVM is not an abbreviation and is the full name of the project. Historically it used to be the abbrevation for "Low-level virtual machine"} compiler infrastructure. Clang is in fact a subproject of the LLVM project. Both Clang and LLVM were originally developed by Chris Lattner.\cite{lattner} \\

The Clang compiler is licensed under the University of Illinois/NCSA License.\cite{clang-policy} This is a fairly permissive license, comparable to the BSD license. It is more permissive than a GPL-style copyleft license, as it does not require publishing changes made to the original source code. The redistributed software only needs to contain the original licensing text and must 
not be endorsed using the names or affiliations of the original authors. Furthermore, no warranty for correct and safe execution is given.\cite{illinois-license} \\

Being part of the LLVM project, Clang is also governed by the non-profit LLVM foundation.\cite{llvm-foundation} The board of the foundation however typically does not decide on low-level code changes and patches. All low-level technical aspects are directly handled by the community.\cite{clang-policy} \\

Contributions from the community are accepted traditionally via a developer mailing list, similar as described for Linux in the previous section. Everyone can submit patches to this mailing list and comment on other contributions. However, only people with write access to the repository can approve patches and merge them into the main repository. Patches are typically reviewed before being committed to the repository. Post-commit review also is an option for trusted maintainers, which allows for more rapid development. In order to obtain write access for the repository, one has to have a track record of high quality contributions and send an application for write access to Chris Lattner, the original creator of Clang. \\

To guarantee that all code is reviewed even in the case of post-commit review, there is a code owner associated to each piece of code in the repository. Code owners are responsible to ensure quality control for their respective area of the code base. Additionally, quality is enforced via detailed coding standards, unit tests and automated nightly builds. \\

As Clang was originally started by Apple, it is no surprise that Apple is still heavily involved in its development. Chris Lattner is still an active member of the LLVM foundation board and also still works for Apple.\cite{lattner} Furthermore, at least the five people with the most commits to the Clang project are all Apple employees.\cite{clang-commits} On the board of the LLVM foundation, there are also employees of ARM, Google and Qualcomm as well as a few people from academia.\cite{llvm-board} The LLVM foundation is sponsored by Apple, Google, Facebook, Intel and a few smaller companies.\cite{llvm-sponsors} \\

Despite the strong industry involvment, it seems that the development of Clang is mostly driven by the community, and less controlled by the LLVM board and the involed companies. According to the official development policy, conflicts should be resolved by discussion via the appropriate mailing list.\cite{clang-policy} Given the permissive license of Clang, it is very likely that some companies maintain an internal version of Clang, which is more tailored towards their respective needs. The most prominent example is Apple, where many new compiler features are first developed internally before they are potentially merged into the open-source Clang repository. The Clang version distributed with Apple's XCode IDE is not the open-source version. \cite{Clang in XCODE}
 This lessens the chance of big conflicts in the future development of Clang. 

%%% Local Variables:
%%% mode: latex
%%% TeX-master: "foss-governance"
%%% End:

