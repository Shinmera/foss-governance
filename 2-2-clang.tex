\subsection{Clang}{Delio Vicini}

Clang is an open-source compiler for C, C++, Objective-C, Objective-C++ and CUDA. The clang compiler project was started by Apple in 2005 in order to replace the GNU Compiler Collection (GCC). The original motivation for the Clang project was to have a fast, modular compiler which is easier to extend
than GCC and also integrates well with integrated development environments (IDEs). \cite{clang-motivation} Technically, the Clang compiler builds on the LLVM\footnote{LLVM is not an abbreviation and is the full name of the project.} compiler infrastructure. 

The Clang compiler is licensed under the University of Illinois/NCSA License. \cite{clang-policy} This is a fairly permissive license, comparable to the BSD license. It is more than the GPL license, in that it does not require publishing changes made to the original source code. \cite{illinois-license}


----
how contributions are handled, reviewed, and submitted (version control)

subsystems and maintainers, how actual decisions are taken

company: involved how and why \\

how do companies affect the decisions: board of directors?\\


conclision


%%% Local Variables:
%%% mode: latex
%%% TeX-master: "foss-governance"
%%% End:

