\subsection{Clang}{Delio Vicini}

Clang is an open-source compiler for C, C++, Objective-C, Objective-C++ and CUDA. The clang compiler project was started by Apple in 2005 in order to replace the GNU Compiler Collection (GCC). The original motivation for the Clang project was to have a fast, modular compiler which is easier to extend
than GCC and also integrates well with integrated development environments (IDEs). \cite{clang-motivation} Technically, the Clang compiler builds on the LLVM\footnote{LLVM is not an abbreviation and is the full name of the project.} compiler infrastructure. Clang is in fact a subproject of the LLVM project.

The Clang compiler is licensed under the University of Illinois/NCSA License. \cite{clang-policy} This is a fairly permissive license, comparable to the BSD license. It is more permissive than a GPL-style license, as it does not require publishing changes made to the original source code. The redistributed software only needs to contain the original licensing text and must 
not be endorsed using the names or affiliations of the original authors. Furthermore, no warranty for correct and safe execution is given. \cite{illinois-license}

Being part of the LLVM project, Clang is also governed by the non-profit LLVM foundation.\cite{llvm-foundation} The board of the foundation however typically does not decide on low-level code changes and patches. 

Contributions from the community are handled traditionally via a developer mailing list. Everyone can submit patches to this mailing list and comment on other contributions. However, only people with write access to the repository can approve patches. Patches are typically reviewed before being committed to the repository. Post-commit review also is an option for trusted maintainers, which allows more rapid development. In order to obtain write access for the repository, one has to have a track record of high quality contributions and send an application for write access to Chris Lattner, the original creator of both LLVM and Clang. 

To guarantee that all code is reviewed even in the case of post-commit review, there is a code owner associated to each piece of code in the repository. Code owners are responsible to ensure quality control for their respective area of expertise. Additionally, quality is enforced via detailed coding standards, tests and automated nightly builds.

As Clang was originally started by Apple, it is no surprise that Apple is still heavily involved in its development. Chris Lattner, who both started the LLVM and Clang project, is still an active member of the LLVM foundation board. Furthermore, at least the five people with the most commits are all Apple employees. \cite{clang-commits}. On the board of the LLVM foundation, there are also employees of ARM, Google and Qualcomm as well as people from academia.\cite{llvm-board} The LLVM foundation is also sponsored Apple, Google, Facebook, Intel and a few other companies.\cite{llvm-sponsors}

However, it seems that the development of Clang is mostly driven by the community, and less controlled by the LLVM board. According to the development policy, conflicts should be resolved by discussion via the appropriate mailing list. Given the permissive license of Clang, it is very likely that some companies maintain an internal version of Clang, which is more tailored towards their respective needs.

In conclusion, it seems that there is a strong involvement of various large IT companies. However, the development of Clang seems mostly community controlled. In the official LLVM/Clang developer policy, there is a strong emphasis on community discussions. With the concept of code owners, there are individuals personally responsible for all parts of the code. 


%%% Local Variables:
%%% mode: latex
%%% TeX-master: "foss-governance"
%%% End:

