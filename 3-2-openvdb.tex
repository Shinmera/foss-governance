\subsection{OpenVDB}{Delio Vicini}

OpenVDB is an open-source C++ library to handle sparse, time-varying volumetric data sets for computer graphics applications. The format was originally developed at DreamWorks Animation and used to render clouds in "Puss in Boots" and subsequently open-sourced in 2012.\cite{Museth.2013} The authors of the library even received a technical achievment academy award for OpenVDB.\cite{openvdb-about} The abbreviation VDB stands for "Volumetric, Dynamic grid that
shares several characteristics with B+trees".\cite{Museth.2013} The core of OpenVDB is an efficient data-structure to handle large sparse volumetric datasets. Additionally the library supports a variety of operations to process volumetric data and also provides basic visualization capabilities. Since its open-source release in 2012, the OpenVDB file format has become one of the standard formats for volume data in the computer graphics industry and is supported by major tools such as Arnold, Renderman and Houdini.\cite{openvdb-about} Having a broad tool support most likely was a driving motiation for Dreamworks to release the library to the public.


% Community
OpenVDB is licensed under the Mozilla Public License (MPL) Version 2.0, which is a compromise between a permissive license such as BSD and GPL. In practice this means, that modifications to MPL 2.0 licensed source code must be made publicly available, but not the whole distributed program.\cite{mpl-faq}

Despite its widespread use, the development of OpenVDB is still primarily lead by Dreamworks. On the OpenVDB Github repository, only a handful of commits do not come from Dreamworks employees.\cite{openvdb-contribs} In order to participate in the development, one has to sign an "Open Source Software Grant and Contributor License Agreement".\cite{openvdb-agree} In this agreement, a contributor has to grant Dreamworks Animation copyright and patent licenses to freely use any contributions. 


% code
The code quality is ensured by rigorous coding standards.\cite{openvdb-code} As the project is rather small and the amount of contributions from the community are not too many, there does not seem to be a very formal code reviewing mechanism. Technically, contributions are accepted as pull requests on Github. It is up to Dreamworks to decide which contributions get included into the repository. It is not entirely clear from the outside how much ressources are still dedicated to further the development of OpenVDB. As of writing this document, there are some open pull requests on Github which are a few weeks old and did not yet get any reaction from the maintainers at Dreamworks.

% conflicts
Since OpenVDB implements a file format, there is not much room for different versions. This could potentially give rise to conflict. However, since the community is very small we could not find any evidence of bigger conflicts and also no official guidelines how these should be handled. In any case, Dreamworks has the final decision what gets into the main repository.


%%% Local Variables:
%%% mode: latex
%%% TeX-master: "foss-governance"
%%% End:
